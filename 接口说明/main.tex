%!TeX program = xelatex
\documentclass[12pt,hyperref,a4paper,UTF8]{ctexart}
\usepackage{NUDTReport}
\usepackage{subcaption}
\usepackage{minipage-marginpar}


\usepackage{listings}
\usepackage{xcolor}
\definecolor{mygreen}{rgb}{0,0.6,0}
\definecolor{mygray}{rgb}{0.5,0.5,0.5}
\definecolor{mymauve}{rgb}{0.58,0,0.82}

\lstset{ %
	backgroundcolor=\color{white},   % choose the background color; you must add \usepackage{color} or \usepackage{xcolor}
	basicstyle=\footnotesize,        % the size of the fonts that are used for the code
	breakatwhitespace=false,         % sets if automatic breaks should only happen at whitespace
	breaklines=true,                 % sets automatic line breaking
	captionpos=bl,                   % sets the caption-position to bottom
	commentstyle=\color{mygreen},    % comment style
	deletekeywords={...},            % if you want to delete keywords from the given language
	escapeinside={\%*}{*)},          % if you want to add LaTeX within your code
	extendedchars=true,              % lets you use non-ASCII characters; for 8-bits encodings only, does not work with UTF-8
	frame=single,                    % adds a frame around the code
	keepspaces=true,                 % keeps spaces in text, useful for keeping indentation of code (possibly needs columns=flexible)
	keywordstyle=\color{blue},       % keyword style
	%language=Python,                % the language of the code
	morekeywords={*,...},            % if you want to add more keywords to the set
	numbers=left,                    % where to put the line-numbers; possible values are (none, left, right)
	numbersep=5pt,                   % how far the line-numbers are from the code
	numberstyle=\tiny\color{mygray}, % the style that is used for the line-numbers
	rulecolor=\color{black},         % if not set, the frame-color may be changed on line-breaks within not-black text (e.g. comments (green here))
	showspaces=false,                % show spaces everywhere adding particular underscores; it overrides 'showstringspaces'
	showstringspaces=false,          % underline spaces within strings only
	showtabs=false,                  % show tabs within strings adding particular underscores
	stepnumber=1,                    % the step between two line-numbers. If it's 1, each line will be numbered
	stringstyle=\color{orange},      % string literal style
	tabsize=2,                       % sets default tabsize to 2 spaces
	%title=myPython.py               % show the filename of files included with \lstinputlisting; also try caption instead of title
}

%%-------------------------------正文开始---------------------------%%
\begin{document}
	%	\tableofcontents
	%	\thispagestyle{empty} % 首页不显示页码
	%	\newpage
	\begin{Huge}
		\begin{center}
			接口说明 v1.0
		\end{center}
	\end{Huge}
	\begin{center}
	欧阳嘉鸿
	
		\today
	\end{center}
	GetDeviceData方法是DeviceDataSource类的一个公用方法,用于获取设备数据。该方法有多个重载版本,分别用于获取string和double类型的数据。其中,用于获取double类型数据的重载版本接受一个DoubleKey类型的参数,并返回一个double?类型的值,即可能返回null的double类型值。
	
	GetDeviceData方法(double重载)的实现细节:
	
	1. 方法首先获取传入DoubleKey对象的Key属性,该属性是一个字符串,表示要获取的数据的键。
	
	2. 接着,方法检查DeviceData字典(假设是类的一个成员,用于存储设备数据)是否包含该键。
	
	3. 如果包含,方法会尝试从字典中获取对应键的值,并将其转换为double类型。
	
	4. 如果转换成功,方法返回该double值;如果转换失败(即值不是double类型),方法返回null。
	
	5. 如果DeviceData字典不包含该键,方法同样返回null。
	\begin{lstlisting}[language=c++,title=注:GetDeviceData方法(double重载)的定义]
		/// <summary>
		/// 获得数据
		/// </summary>
		/// <param name="dataKey"></param>
		/// <returns></returns>
		public double? GetDeviceData(DoubleKey dataKey)
		{
			string key = dataKey.Key;
			if (DeviceData.ContainsKey(key))
			{
				object o = DeviceData[key];
				if (o is double)
				{
					return (double)o;
				}
				return null;
			}
			else
			{
				return null;
			}
		}
	\end{lstlisting}
	
	在程序外部,可以通过创建DeviceDataSource类的实例(如BWT901BLE)并调用其GetDeviceData方法来获取设备数据。例如,要获取当前时刻的\(x\)轴加速度值,可以传入WitSensorKey.AccX作为参数。WitSensorKey是一个包含设备数据关键字的类,AccX是其一个静态成员,表示\(x\)轴加速度数据的关键字。
	
	在我们项目的编程中,使用BWT901BLE作为\emph{传感器实例}的名称,我们通过使用\emph{方法访问操作符}(即“.”)访问DeviceDataSource实例的GetDeviceData方法,利用上述代码中的语句即可获得当前时刻的\(x\)轴加速度值。其中WitSensorKey.AccX为需要传入GetDeviceData方法的关键字,在使用的时候不需要管它的定义。
	\begin{lstlisting}[language=c++,title=注:使用GetDeviceData方法的示例]
		double AccX = BWT901BLE.GetDeviceData(WitSensorKey.AccX);
	\end{lstlisting}

	如下代码中还列出了所有可能用到的数据关键字及其对应的数据类型。这些关键字包括芯片时间、加速度(\(x\)、\(y\)、\(z\)轴和矢量和)、角速度(\(x\)、\(y\)、\(z\)轴和矢量和)、角度(\(x\)、\(y\)、\(z\)轴)、温度、四元数(0、1、2、3)、版本号、序列号和电量等。通过传入这些关键字作为参数,可以获取相应的设备数据。
	\begin{lstlisting}[language=c++]
		// 芯片时间
		string ChipTime =BWT901BLE.GetDeviceData(WitSensorKey.ChipTime);
		// 加速度X
		double AccX =BWT901BLE.GetDeviceData(WitSensorKey.AccX);
		// 加速度Y
		double AccY =BWT901BLE.GetDeviceData(WitSensorKey.AccY);
		// 加速度Z
		double AccZ =BWT901BLE.GetDeviceData(WitSensorKey.AccZ);
		// 加速度矢量和
		double AccM =BWT901BLE.GetDeviceData(WitSensorKey.AccM);
		// 角速度
		double AsX =BWT901BLE.GetDeviceData(WitSensorKey.AsX);
		// 角速度
		double AsY =BWT901BLE.GetDeviceData(WitSensorKey.AsY);
		// 角速度
		double AsZ =BWT901BLE.GetDeviceData(WitSensorKey.AsZ);
		// 角速度Z矢量和
		double AsM =BWT901BLE.GetDeviceData(WitSensorKey.AsM);
		// 角度X
		double AngleX =BWT901BLE.GetDeviceData(WitSensorKey.AngleX);
		// 角度Y
		double AngleY =BWT901BLE.GetDeviceData(WitSensorKey.AngleY);
		// 角度Z
		double AngleZ =BWT901BLE.GetDeviceData(WitSensorKey.AngleZ);
		// 温度
		double T =BWT901BLE.GetDeviceData(WitSensorKey.T);
		// 四元数0
		double Q0 =BWT901BLE.GetDeviceData(WitSensorKey.Q0);
		// 四元数1
		double Q1 =BWT901BLE.GetDeviceData(WitSensorKey.Q1);
		// 四元数2
		double Q2 =BWT901BLE.GetDeviceData(WitSensorKey.Q2);
		// 四元数3
		double Q3 =BWT901BLE.GetDeviceData(WitSensorKey.Q3);
		// 版本号
		string VersionNumber =BWT901BLE.GetDeviceData(WitSensorKey.VersionNumber);
		// 序列号
		string SerialNumber =BWT901BLE.GetDeviceData(WitSensorKey.SerialNumber);
		// 电量
		double PowerPercent =BWT901BLE.GetDeviceData(WitSensorKey.PowerPercent);
	\end{lstlisting}
	
	在具体使用过程中,首要任务是创建一个适宜的容器,用于存储一段时间内采集的数据,随后对这些数据进行分段处理。鉴于传感器以200Hz的频率获取数据,编程时需特别考虑容器的大小和计算效率,这对数据处理至关重要。同时,传感器的数据获取频率直接影响到分段处理的策略。因此,在编程过程中,必须兼顾实际需求与资源限制,以确保代码的可行性和可维护性。
	
\end{document}